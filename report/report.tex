\documentclass{tpk4170report}

\title{Title of Report}
\author{Author(s)}

\usepackage{blindtext}

\addbibresource{references.bib} 

\begin{document}

\maketitle

\frontmatter

\chapter*{Preface}

\tableofcontents
\listoffigures
\listoftables

\mainmatter








\chapter{Introduction}
\label{chap:Introduction}

\blindmathtrue
\blindtext

This is a citation~\cite{McCarthy2011}. \blindtext

See Table~\ref{table:floating-table} for a floating table and
\eqref{eq:equation-example} for an equation.
\begin{align}
  \label{eq:equation-example}
  y = ax + b 
    = cz + d 
\end{align}

\begin{table}
  \centering
  \begin{tabular}{l|lll}
    a& b& c &d \\
    \hline
    a& b& c &d \\
    a& b& c &d \\
    a& b& c &d
  \end{tabular}
  \caption{This is a floating table}
  \label{table:floating-table}
\end{table}











\chapter{Here Comes Chapter 2}

\Blindtext

\begin{figure}
  \centering
  \includegraphics[width=0.5\textwidth]{hovedlogo} 
  \caption{NTNU logo}
  \label{fig:logo2}
\end{figure}

\section{Title of section 2}








\chapter{Here Comes Chapter 3}

\Blindtext

\begin{figure}
  \centering
  \includegraphics[width=0.5\textwidth]{hovedlogo} 
  \caption{NTNU logo}
  \label{fig:logo}
\end{figure}





\chapter{Conclusion}

\blindtext[4]



\printbibliography[title=References]

\appendix
\chapter{Name of Appendix} 

\section{This is a section}

\end{document}